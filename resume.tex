% !TEX program = xelatex

\documentclass{resume}
% \usepackage{zh_CN-Adobefonts_external} % Simplified Chinese Support using external fonts (./fonts/zh_CN-Adobe/)
\usepackage{zh_CN-Adobefonts_internal} % Simplified Chinese Support using system fonts
% \setCJKmainfont[ItalicFont={AR PL UKai CN}]{AR PL UMing CN} %设置中文默认字体
% \setCJKsansfont{WenQuanYi Zen Hei} %设置文泉驿正黑字体作为中文无衬线字体
% \setCJKmonofont{WenQuanYi Zen Hei Mono} %设置文泉驿等宽正黑字体作为中文打字机字体
% \usepackage{fancyhdr}
\begin{document}
% \fancyhead{}
\pagenumbering{gobble} % suppress displaying page number
\pagestyle{plain}
\name{张晓云}

% {E-mail}{mobilephone}{homepage}
\basicContactInfo{elendil@stu.xjtu.edu.cn}{(+86) 18991952901}
% {E-mail}{mobilephone}
% \basicContactInfo{xxx@yuanbin.me}{(+86) 131-221-87xxx}
 
\section{\faGraduationCap\ 教育经历}
\datedsubsection{\textbf{西安交通大学}}{2014/9 -- Present}
\textit{计算机科学与技术}专业
\begin{itemize}
  \item GPA: 3.65
  \item 排名: 4/157
\end{itemize}
\datedsubsection{\textbf{National University of Singapore}}{2016/8 -- 2016/12}
\textit{School of Computing}
\begin{itemize}
  \item 交换生
\end{itemize}

\section{\faUsers\ 项目经验}
\datedsubsection{\textbf{EDEN Lab}}{2015/8 -- Present}
\role{实验室成员}{}
\begin{itemize}
  \item 使用\textit{SIR}模型模拟\textit{EBSN}网络中的事件传播,并据此建立推荐模型,达到了60\%以上的成功率。
  \item 使用\textit{lasso}回归对事件描述建模,并建立了关于事件参与人数和事件描述的混合模型,证明了事件描述可以显著的影响事件参与人数。
\end{itemize}

\datedsubsection{\textbf{微软互联网工程院 苏州分院}}{2017/7 -- 2017/10}
\role{实习生}{}
\begin{itemize}
  \item 设计并实现了一个日志分析工具,用于发现并分类设备OOBE过程中出现的错误
  \item 预处理使用了\textit{tf-idf},分类器使用了\textit{linear svm},并使用了\textit{PCA}来发现未知错误
  \item 准确率达到了98\%,至今仍运行良好
\end{itemize}

\datedsubsection{\textbf{National University of Singapore}}{2016/8 -- 2016/12}
\begin{itemize}
  \item 使用\textit{Ocaml}为\textit{mini-go}(\textit{Go}语言的子集)实现了一个编译器,同时使用\textit{Ocaml}实现了一个虚拟机用来运行编译后的程序。
\end{itemize}

\section{\faHeartO\ 获得奖励}
\begin{itemize}
  \item 国家奖学金 \textit{3/157} \hfill 2016/10
  \item 彭康奖学金 \textit{5/157} \hfill 2015/10
\end{itemize}
\section{\faCogs\ 技能}
\begin{itemize}[parsep=0.5ex]
  \item 编程语言: Python > ( C \& Rust) > C++ $\gg$ Java \& Go
  \item 平台: Linux $\approx$ Windows
  \item 英语:TOFEL 105
\end{itemize}
\end{document}
